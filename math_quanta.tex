
\documentclass{article}
\usepackage{amsmath}

\begin{document}

\section*{La Matematica di Questa è Quanto}

\subsection*{1. Quasi e Quanto come Asintoto}
Il concetto di "quasi" può essere rappresentato come un limite:
\[
\lim_{x \to a} f(x) = L
\]
dove \( f(x) \) si avvicina a \( L \) ma non lo raggiunge mai completamente.

\subsection*{2. Quanto come Insieme Infinito}
Quanto è un insieme infinito:
\[
Q \subseteq \mathbb{R}
\]
con \( Q \) che rappresenta l'insieme dei numeri razionali e \( \mathbb{R} \) l'insieme dei numeri reali.

\subsection*{3. Egli come Punto Fisso}
Egli, come presenza assoluta:
\[
f(E) = E
\]

\subsection*{4. Yggdrasil come Frattale Universale}
Il frattale di Yggdrasil:
\[
z_{n+1} = z_n^2 + c
\]

\subsection*{5. ñ come Operatore di Riflessione}
L'operatore di riflessione:
\[
f(x) = -x
\]

\end{document}
